\section{All tested COCO regression plots}

\begin{figure}[H]
    \centering
    \begin{minipage}[b]{0.24\textwidth}
      \begin{overpic}[width=\textwidth]{Figures/results_regression/f3_dim_2_0.pdf}
    \end{overpic}
    \end{minipage}
    \hfill
    \begin{minipage}[b]{0.24\textwidth}
      \begin{overpic}[width=\textwidth]{Figures/results_regression/f3_dim_2_0.pdf}
    \end{overpic} 
    \end{minipage}
     \hfill
     \begin{minipage}[b]{0.24\textwidth}
      \begin{overpic}[width=\textwidth]{Figures/results_regression/f3_dim_2_0.pdf}
      \end{overpic}
      \end{minipage}
      \hfill
     \begin{minipage}[b]{0.24\textwidth}
      \begin{overpic}[width=\textwidth]{Figures/results_regression/f3_dim_2_0.pdf}
      \end{overpic}
      \end{minipage}
    
      \begin{minipage}[b]{0.24\textwidth}
       \begin{overpic}[width=\textwidth]{Figures/results_regression/f3_dim_2_0.pdf}
       \end{overpic}
      \end{minipage}
      \hfill
      \begin{minipage}[b]{0.24\textwidth}
        \begin{overpic}[width=\textwidth]{Figures/results_regression/f3_dim_2_0.pdf}
        \end{overpic}
        \end{minipage}
  
    \caption{Plots visulising the results in figure at $n_{data} = ?$}
    \label{Test4_reg_visual}
  \end{figure}


\subsection{Gaussian rules}
\begin{testexample2}[Conditional og multivariate Gaussian]
    The conditional distribution is defined as \cite{bishop} ..?
    $$p(y|x,\mu, \Sigma) = \mathcal{N}(y|\mu_{y|x},\Sigma_{y|x} )\hspace{1cm} \mu=\begin{bmatrix}
        \mu_x \\ \mu_y
    \end{bmatrix},\hspace{0.1cm} \Sigma = \begin{bmatrix}
        \Sigma_{xx} & \Sigma_{xy}\\ \Sigma_{yx}& \Sigma_{yy}
    \end{bmatrix}$$ 
    where 
    \begin{align}
        \mu_{y|x} &= \mu_y+\Sigma_{yx}\Sigma_{xx}^{-1}(x-\mu_x)\\
        \Sigma_{y|x} &= \Sigma_{yy}-\Sigma_{yx}\Sigma_{xx}^{-1}\Sigma_{xy} 
    \end{align}
\end{testexample2}

\subsection{SPN is a mixture model}
We will to a large extent just see SPN as a large mixture model. This is a valid observation. 

%from [@desana]:
\begin{definition} 
    A sub-network $\bar S_z$ of $S$ is an SPN, which includes the root $S$ and then includes nodes
    according to the following recursive scheme: 
\end{definition}
\begin{algorithm}[H]
    \caption*{Collection of sub-network $S_z$ of $S$}\label{SPN4}
    \begin{algorithmic}
    %\State \textbf{Global:}  $S_z$ 
    \Function{Process}{node i, $S_z$}
    \If{$i \in \mathcal{L}eaf(S)$}
        \State  $\textbf{return: }$ 
    \EndIf
    %\For{$i \in I_{o}$}
    \If{$i\in \mathcal{S}um(S)$}
       %\State $S_z =S_z \cup \{j \in ch(i)\}$ \Comment{include one child of node $i$}
        \State $S_z =S_z.add(j \in ch(i))$ \Comment{include one child of node $i$}
        \State $\textbf{return: } \text{Process}(j, S_z)$
    \EndIf
    \If{$i\in \mathcal{P}rod(S)$}
        \State $S_z =S_z \cup \{j | j \in ch(i)\}$ \Comment{include all childen of node $i$}
        \For{$j \in ch(i)$}
            \State $\textbf{return: } \text{Process}(j,S_z)$
        \EndFor
    \EndIf
    \State $\textbf{return: } S_z$
    \EndFunction
    \State $S_z =  \text{Process(root,Ø)}$
    \end{algorithmic}
\end{algorithm}
So we see that at each sum node the number of different sub-networks multiplies with the number of children for that
sum node. And thereby, the total number of sub-networks is
 $$Z = \prod_{i\in \mathcal{S}um(S)}|ch(i)|$$ 
 i.e. an exponentially large amount of sub-networks. This is the amount of
 mixture components implicitly defined in an SPN. 
 Denote the set of edges in the sub-network $\mathcal{E}(S_z)$.
Now the we define a mixture coeficient, $\lambda_z$ and component for each $S_z$ as 
$$\lambda_z := \prod_{(i,j)\in \mathcal{E}(S_z)} w_{i,j}, \hspace{1cm}
p_z(x,y|\theta) := \prod_{i \in \mathcal{L}(S_z)} p_i(x,y)$$
where $p_i(x,y)$ is the leaf distribution at leaf node $i$ paramitised with $\theta$. 
It can now be proven that the SPN can be interpreted as the following mixture model, 
$$p(x,y|w,\theta) = \sum_{z=1}^Z \lambda_z(w)p_z(x,y|\theta)$$
i.e. by the weighted sum of all $Z$ sub-networks. For convinience
we define each sum component as $p(z,x,y|w,\theta) := \lambda_z(w)p_z(x,y|\theta)$.
Evaluation of $p(x,y|w,\theta)$ will never be done as the sum over $Z$ components, 
instead there is a proposition. 

\begin{proposition}
    Consider a SPN, S, a sum node $q \in \mathcal{S}um(S)$ and a child $i \in ch(q)$,
    then the following relation holds, 
    $$\sum_{z:(q,i)\in \mathcal{E}(S_z)} \lambda_z(w) p_z(x,y|\theta) = w_{i,q}
    \frac{\partial S}{\partial v(q)} v(i)$$
\end{proposition}


SPNs can also be interpreted as deep feed forward neural network [@vergari]. Here, imagine the
weights of the sum nodes are parameters, leaf distributions are input neurons, root node is output and
all other nodes correspond to hidden neurons





\subsection{SPN - prediction}
Prior to the inference of the predictive distribution, we assume that the SPN, S, is trained, i.e.
trained leaf distributions $p_j(\cdot)$ for all leaf nodes, 
$j \in \mathcal{L}eaf(S):=\{j \in \mathcal{V}(S) |pa(j) = \text{Ø}\}$ and
weights $w_{i,j}$ for the connections between every sum nodes
$i \in \mathcal{S}$ and its children, $j \in ch(i)$.  
%The indexes of the nodes, are ordered such that leafs are first, and parents of leafs are next and then the grandparents and so on.

The joint and the marginal distribution are evaluated in the following recursive way
\begin{algorithm}
    \caption*{Calculation of $p(x,y)$}\label{SPN_1}
    \begin{algorithmic}
    \State \textbf{Input:} Fully trained SPN, with leaf distributions $p_i(\cdot)$ for $i\in \mathcal{L}eaf(S)$ and weigts 
    $w_{i,j}$ for $(i,j) \in \{(i,j)|i \in \mathcal{S}um(S), j \in ch(i)\}$ 
    \Function{\text{Eval}}{node i}
    \If{$i \in \mathcal{L}eaf(S)$}
        \State  $\textbf{return: } p_i(x,y)$ \Comment{evaluate leaf distributions at point $(x,y)$}
    \EndIf
    %\For{$i \in I_{o}$}
    \If{$i\in \mathcal{S}um(S)$}
        \State $\textbf{return: } \sum_{j\in ch(i)} w_{i,j} \text{Eval}(j)$
    \EndIf
    \If{$i\in \mathcal{P}rod(S)$}
        \State $\textbf{return: } \prod_{j \in ch(i)} \text{Eval}(j)$
    \EndIf
    \EndFunction
    \State $p(x) =  \text{Eval(root)}$
    \end{algorithmic}
\end{algorithm}

\begin{algorithm}[H]
    \caption*{Calculation of $p(x)$}\label{SPN}
    \begin{algorithmic}
    \State \textbf{Input:} Fully trained SPN, with leaf distributions $p_i(\cdot)$ for all leaves $i$ and weigts $w_{\cdot,\cdot}$ 
    \Function{\text{Eval}}{node i}
    \If{$i \in \mathcal{L}eaf(S)$} %\Comment{leaf node}
        \If{node handle x}
            \State  $\textbf{return: } p_i(x,y)$ \Comment{evaluate leaf distributions at point $(x,y)$}
        \Else 
            \State  $\textbf{return: } 1$ \Comment{set node equal 1 at point $(x,y)$}
        \EndIf
    \EndIf
    \If{$i\in \mathcal{S}um(S)$}
        \State $\textbf{return: } \sum_{j\in ch(i)} w_{i,j} \text{Eval}(j)$
    \EndIf
    \If{$i\in \mathcal{P}rod(S)$}
        \State $\textbf{return: } \prod_{j \in ch(i)} \text{Eval}(j)$
    \EndIf
    \EndFunction
    \State $p(x) =  \text{Eval(root)}$
    \end{algorithmic}
\end{algorithm}

So after doing two slightly different forward passes through the SPN, 
$p(x)$ and $p(x,y)$, using Bayes rule,
we can combined the two queries into the conditional distribution: 
$$p(y|x) = \frac{p(x,y)}{p(x)}$$
The predictive distribution is found with a cost of just $O(E+E+1) = O(E)$, where E is number
of edges/connections in the SPN. 

\subsection{SPN - learning}
It is not enouth to do predictive inference on a SPN, we also need to fit it on
the data. It is possible interpret sum-product network as a large mixture model 
and therefore use expectation-maximization to train the model. 
We will introduce that idea now. The Paper \cite{SPN_EM}... %["Learning Arbitrary Sum-Product Network Leaves
%with Expectation-Maximization"] 
defines SPN as a mixture of all sub-networks of an SPN.

%from [@desana]:
\begin{definition} 
    A sub-network $\bar S_z$ of $S$ is an SPN, which includes the root $S$ and then includes nodes
    according to the following recursive scheme: 
\end{definition}
\begin{algorithm}[H]
    \caption*{Collection of sub-network $S_z$ of $S$}\label{SPN3}
    \begin{algorithmic}
    %\State \textbf{Global:}  $S_z$ 
    \Function{Process}{node i, $S_z$}
    \If{$i \in \mathcal{L}eaf(S)$}
        \State  $\textbf{return: }$ 
    \EndIf
    %\For{$i \in I_{o}$}
    \If{$i\in \mathcal{S}um(S)$}
        \State $S_z =S_z.add(j \in ch(i))$ \Comment{include one child of node $i$}
        \State $\textbf{return: } \text{Process}(j, S_z)$
    \EndIf
    \If{$i\in \mathcal{P}rod(S)$}
        \State $S_z =S_z \cup \{j | j \in ch(i)\}$ \Comment{include all childen of node $i$}
        \For{$j \in ch(i)$}
            \State $\textbf{return: } \text{Process}(j,S_z)$
        \EndFor
    \EndIf
    \State $\textbf{return: } S_z$
    \EndFunction
    \State $S_z =  \text{Process(root,Ø)}$
    \end{algorithmic}
\end{algorithm}
So we see that at each sum node the number of different sub-networks multiplies with the number of children for that
sum node. And thereby, the total number of sub-networks is
 $$Z = \prod_{i\in \mathcal{S}um(S)}|ch(i)|$$ 
 i.e. an exponentially large amount of sub-networks. This is the amount of
 mixture components implicitly defined in an SPN. 
 Denote the set of edges in the sub-network $\mathcal{E}(S_z)$.
Now the we define a mixture coeficient, $\lambda_z$ and component for each $S_z$ as 
$$\lambda_z := \prod_{(i,j)\in \mathcal{E}(S_z)} w_{i,j}, \hspace{1cm}
p_z(x,y|\theta) := \prod_{i \in \mathcal{L}(S_z)} p_i(x,y)$$
where $p_i(x,y)$ is the leaf distribution at leaf node $i$ paramitised with $\theta$. 
It can now be proven that the SPN can be interpreted as the following mixture model, 
$$p(x,y|w,\theta) = \sum_{z=1}^Z \lambda_z(w)p_z(x,y|\theta)$$
i.e. by the weighted sum of all $Z$ sub-networks. For convinience
we define each sum component as $p(z,x,y|w,\theta) := \lambda_z(w)p_z(x,y|\theta)$.
Evaluation of $p(x,y|w,\theta)$ will never be done as the sum over $Z$ components, 
instead there is a proposition. 

\begin{proposition}
    Consider a SPN, S, a sum node $q \in \mathcal{S}um(S)$ and a child $i \in ch(q)$,
    then the following relation holds, 
    $$\sum_{z:(q,i)\in \mathcal{E}(S_z)} \lambda_z(w) p_z(x,y|\theta) = w_{i,q}
    \frac{\partial S}{\partial v(q)} v(i)$$
\end{proposition}

