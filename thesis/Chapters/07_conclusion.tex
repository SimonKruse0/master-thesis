\chapter{Discussion}
In the following, we will discuss the results from the previous chapter and the
surrogate models in general. 

%Tests done correctly?
From the BO results, we found that the GP showed overall best performance on the four selected BBOB
problems in higher dimensions and Test1, while SPN won Test2, Random search won Test3, and BNN won
Test4. Now, the question arises about how valid the results are. Looking at Figure
\ref{BayesOpt_all} in the appendix we see how noisy the 20 BO paths are --- sometimes a method just
luckily chooses a point close to the minimum. This is the reason why we average across the 20 seeded
BO paths, as it will quantify the models general performance on the problem. For more robust
results, we should have included more optimization paths, however, due to time and resources only 20
experiments were conducted. Another improvement, would be to include more BO iterations. While all
1D tests are minimized nicely, most of the higher dimensional tests are not; naturally, regression
in higher dimensional spaces need more training data point for good performance. The test conducted
on BBOB with \cite{PhDthesis} used a budget of 100 function evaluations. 

% Important that connection between the Bayesian regression and BO.
In the result section we tried to establish a connection between the performance of Bayesian
optimization and the surrogate model's performance as a regression model. However, the GP which in
general performed well on the BO tests, had a very good predictive power, but also a bad uncertainty
quantification (UQ). The GPs bad UQ was corced by its overconfident about a predictions (i.e. very
small predicitive variance $\sigma(x,\mathcal{D})^2 \approx 0$), however, as long it mean
predicition is good, it is not as important --- in the expecpected improvement, the fraction
$\frac{y_{\min}-\mu_x}{\sigma_x}$ goes to infinity for $\sigma_x \rightarrow 0$, resulting in the
improvement, 
$$EI(x) = (y_{\min}-\mu_x)\Phi\left(\frac{y_{\min}-\mu_x}{\sigma_x}\right)+ \sigma_x
\phi\left(\frac{y_{\min}-\mu_x}{\sigma_x}\right) \approx y_{\min}-\mu_x,$$ since the standard
Gaussian cdf and pdf become 1 and 0, respectively. 

The connection between the mixture models' good performance as regression models (good uncertainty
quantification) and their BO performance was not fair. The predictive distributions of the mixture
models are much richer than the Gaussian approximation used in BO tests. A better connection would
be established if we conducted BO tests using the approximate expected improvement \eqref{aEI}. The
predictive distribution of the Bayesian neural networks is also approximated with a Gaussian. This
is more reasonable as the Bayesian neural networks in the limit are equal to a Gaussian process.
Furthermore, experiments using the approximated expected improvement with the Bayesian Neural
networks were conducted in \cite{PhDthesis}, with no promising results. 

%\todo{Discriminative vs generative models}
In this thesis, we tried using mixture regression models as surrogate models. Especially, when
breaking the GP and BNN's assumptions of Gaussian observation noise, they proved effective. This
class of problems might occur in real-life simulations where the objective function can have
multiple values, which are not just nearby values coursed by measurement noise. If one knew the
"generative story" of the objective function values (e.g. that Test3 jumps between 2 functions), it
might be possible to think about combining more Gaussain proceses in a probalistic graphical model,
i.e. flipping a coin on what GP to use. But in the case where this generative story was not given
mixture regression might indeed be a smart choice. 

Discriminative models are built with this one purpose of regression, whereas generative models
conduct different kinds of probabilistic queries, which are useful. Discriminative models are good
for finding the paterns in data.

\section{Surrogate models}
We also want to discuss the surrogates in more detail,

For the GPs we never discussed the kernel. The Matérn kernel is a common choice for problems with
stationarity assumptions \cite{TakeHumanOutOfLoop}. Especially the non-stationary test4 was not a
good choise for the Matérn kernel. 

The Bayesian neural network training is challenging to know when it has converged, in general the
more samples the better, and the more weights the more complicated model there for more samples. We
chose 500 warm-up and 500 samples, using 4 chains, since we have 4 cpus. Also to make sure the
fitting we sat a very informative prior on a very small observation variance. 

BOHamiANN is in general performing bad. But got a comback for higher dims. It is designed to have a
fast inference, while second to have on pair performance with the GP but with only linearly scaling
inference. We fixed the number of burn-in samples to 1000 and sampled every 40'th of
the next 2000 samples. In the \cite{BOHAMIANN}'s code revealed that they used $100\cdot n_{\text{data}}$ burn-in
samples and sampled every 100'th of the next 10000 samples.

Sum product networks are the motivation why we include deep in this thesis\footnote{Justifying the
thesis title, the thesis does work with deep Bayesian neural networks and GP (which can be
interpreted as infinitely large BNNs).}. It can be shown that SPNs can be interpreted as deep
neural networks \cite{SPN}. The depth of the SPN is constraint to $\log_2 (d)$ where $d$ is the
number of dimensions and, unfortunately, it is most common to have a small number of dimensions in
BO problems. In the Figure \ref{SPN_fig} we found that the SPN is quite constraint, especially due
to the $y$ dimension: Choosing 20 leaf distributions, all the mixture components of the regression
model are stuck with 20 different $y$ - yielding a kind of \textit{discretized regression}. It is
evident that SPN is limited in its expressibility in Figure \ref{Test2_reg_plot}. The success of the
SPN in the results is merely due to the fact that it is defined with MAP and not MLE. 

GMR is a non-bayesian mixture model - it is trained by maximum likelihood. It has
the best predicitive power, however, is also failing a lot. Essentially it is just a bit smarter
version of random search, which indeed migth be better than an overconfident GP in certain problems.

\chapter{Conclusion and further work}
The following chapter concludes the thesis and my work over the last five months. Finally the 
we also look a bit toward further work. 

The aim of the thesis was first to derive and understand the foundational theory of Bayesian
optimization, the different proposed surrogate models and necessary inference methods, and secondly,
we wanted to answer the following research question: 
\begin{itemize}
    \item Can mixture regression models and BNNs be effective surrogate models
    performing better than GPs in some complex BO problems? 
\end{itemize}

While skipping some part of the exhaustive theory, the overall theory coverage was found to be
sufficient for conducting BO experiments with the proposed surrogates. Among more, we derived
Expected improvement in closed form, GP inference in closed form, EM for mixture models and
motivated MCMC. We implemented our own Bayesian NN and kernel density estimator regression 
and adapted the RAT-SPN and GMR into a Bayesian setting. The GP and BOHamiANN was implemented
almost straight from python libraries. 

The results showed that the defined regression models can be used as surrogate models in BO
settings. If they are preferred or not is still an open question. The overall picture found from the
results is that the GP is a prefered model for most problems, while we found a few niche cases which
indicated promise for mixture regression. Results should have been conducted more exhasutivly, to answer the research
question better. 
 


% Connect with the introduction! Hypothesis + the contributions

% We throughout the thesis we had a great focus on understanding and comunicating the theory. With a balance of not
% Going too deep into knowledge. In 

% Results showed that the models can be used as surrogate models. If they are effective or not is still a question. 
% as the results should have been conducted more exhasutivly. 


% This thesis has motivated the relevance for Bayesian optimization, developed different alternative
% surrogate models, i.e. a Bayesian NN and mixture regression models and tested those on some 
% test functions. Among the 4 selected BBOB problems in higher dimensions, it was not possible to show that
% GP was not the preferred model. However, for the 4 small 1D test problems, we found problems, where the GP. 
% is not preferable. To the story is that the developed generative surrogates all have been approximated by
% a Gaussian. While the test could have been more elaborate. 

% First of the all, the hypothesis conducted in the beginning was, 
% \begin{itemize}
%     \item Mixture regression models and BNNs can be effective surrogate models
%     performing better than GPs in some complex BO problems. 
% \end{itemize}
% In the results we did find two problems, where the GP is not a preferred model. This confirms that
% we successfully implemented both BNNs and Mixture regression models in the Bayesian optimization setting.  
% However, the tests were not exhaustive enough to conclude whether or not the new methods, in general, 
% are a prefered model in these cases. 



% The SPN was understood in the context of regression, which to our knowledge is a novel approach. However, 
% when diving into the definition of the SPN it became clear that it was nothing but a discretized regression model. 

% The kernel density estimation is an uncorrelated GMM with a component on each datapoint, this made
% it robust, but also very naive, without the prior this model would provide a local mean prediction
% with mean equal the nearest data point and variance is selected.

% The Gaussian mixture regression was included to make the more intelligent mixture components, which
% would allow for better predictive power, however, without much data the model collapsed into very
% spiky distributions, and the prior was often sat high to help out the uncertainty quantification. 

% The BOHamiANN did a surprisingly good job in the Bayesian optimization (for the larger high-dim problems),
% even though it had a 

% We showed that there is a strong connection between being a good regression model and a good Bayesian optimization
% algorithm. Unfortunately, even though the mixture regression models showed good uncertainty quantification,
% they were approximated by gaussians to perform the fast closed form Expected improvement, therefore the 
% connection is lost. 

\section{Further work}
As mentioned, the mixture models were forcefully approximated by a Gaussian in the BO setting, but
also when testing preditive power with mean instead of mode. Therefore, an obvious next step in
further work is to test the mixture models using approximate expected improvement, using samples
from their actual predictive distributions. Additionally, other types of acquisition functions
should be tried out. 

The investigation of this thesis was conducted to understand what surrogate model to choose.
However, the standard answer of a Bayesian should be: "Why choose?". If there is uncertainty about
them, then we can just average them. It could be interesting to combine GP and mixture regression in 
an ensemble model.

Trying out new or improved methods is an obvious next step. For example, we could try out
flow-models for better generative models. Another idea could be to improve the GP performance with
the choice of the GP kernel, especially, deep kernel learning is an interesting topic
\cite{DeepKernelLearning}. Finally, we could try out fully Bayesian models, i.e. assigning the
models hyperpriors instead of manually setting the hyperparamters or choosing them with crossvalidation. 

% In this paper, we have only considered function approximation problems. Problems
% requiring classication could be handled analogously with the appropriate models. For
% learning classication with a mixture model, one would select examples so as to maximize
% discriminability between Gaussians; for locally weighted regression, one would use a logistic
% regression instead of the linear one considered here (Weisberg, 1985).