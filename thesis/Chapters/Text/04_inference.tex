\chapter{Inference: Prediction and learning}
Inference is the process of computing answers to queries about a probabilistic model after observing data. 
In Bayesian regression, the
query is the predictive distribution, $p(y|x,\mathcal{D})$, as we are interested in the distribution of $y$ given $x$ 
and already observed data, $\mathcal{D}$. 
This often indirectly create the posterior query, $p(\theta|\mathcal{D})$, the probability of model parameters $\theta$ given data
$\mathcal{D}$. Lastly it is also inference, when we train 
a Gaussian mixture model or SPN using the expectation-maximization algorithm (EM), since we are iteratively answering the query
$E_{p(z|\theta^{(k)})}[z|\theta]$.

\section{Exact and approximate inference}
We distinguish between two different ways of inference, exact and approximate inference.
It is \textit{exact inference} when a probabilistic query is calculated exact. It is possible to calculate exact inference on 
the predictive distribution for the Gaussian mixture model, Sum product network, and Gaussian processes. Models which allow 
for exact inference have a powerful advantage over the models with approximate inference since we can guarantee
 the answers to the queries are
correct, however, they are usually also less expressive. It is possible to
make exact inference of SPN, Gaussian Process, and Gaussian Mixture Regression. 
% \begin{itemize}
%     \item SPN
%     \item Gaussian Process
%     \item Gaussian Mixture Regression
% \end{itemize}

When it is not possible to answer a probabilistic query exact, we can approximate the
answer using \textit{approximate inference}. When dealing with complicated and expressive statistical models, exact inference is often
intractable and we need to use approximate inference. Approximate inference 
is a broad category of methods, which includes 
variational inference, Laplace approximation, and Markov chain Monte Carlo (MCMC).
The two Bayesian Neural networks we deal with in this project Bohamiann and Numpyro BNN are
similar regression models, but are infered using two different versions of the MCMC 
method, Hamiltonian Monte Carlo. As it will be revealed later (see result section) 
approximate inference might indeed be flawed and inexact. 
%When dealing with complicated and expressive statistical models, exact inference is often
%intractable and we need to use approximate inference, which might indeed be flawed and 
%inexact. 
% \begin{itemize}
%     \item Bohamiann (Adaptive stochastic MCMC)
%     \item Numpyro Bayesian Neural Network (NUTS)
% \end{itemize}

\begin{table}[H]
    \centering
    \begin{tabular}{l|l|l}
    %\rowcolor[HTML]{C0C0C0} 
    \textbf{Model} & \textbf{Predictive inference} &   \textbf{Learning} \\ \hline
    GP          & Exact $O(n^3)$  & Emperical Bayes\\
    SPN             & Exact $O(E)$ &  EM $O(E)$\\
    Gaussian Mixture Regression & Exact $O(K)$ & EM  \\
    Bohamiann                             & Adaptive stochatic HMC & \\
    Numpyro BNN                           & No U-Turn Sampler & 
    \end{tabular}
    \caption{Overview of inference methods applied on the statistical models 
            used in this project. $E$ is the number of edges in the SPN. $n$ is the number of datapoints. 
            $K \leq n$ is the number of mixture comonents. We will soon learn that for an
            SPN the number of mixture compenets is exponential larger than number of edges
            i.e. $E << K$. In theory MCMC methods samples 
            from true the posterior distribution, and do not need any fitting/learning. 
            }
\end{table}

\section{SPN}
Sum-product networks are generative models, i.e. statistical models of the joint distribution $p(x,y)$. 
We need, however, a disciminative model for regression, i.e. 
a model of the conditional distribution $p(y|x)$. 
SPNs allow for exact inference of the joint distribution 
and any marginalized distribution. These combined queries is sufficient for the
exact predictive posterior. 

\subsection{SPN - prediction}
Prior to the inference of the predictive distribution, we assume that the SPN, S, is trained, i.e.
trained leaf distributions $p_j(\cdot)$ for all leaf nodes, 
$j \in \mathcal{L}eaf(S):=\{j \in \mathcal{V}(S) |pa(j) = \text{Ø}\}$ and
weights $w_{i,j}$ for the connections between every sum nodes
$i \in \mathcal{S}$ and its children, $j \in ch(i)$.  
%The indexes of the nodes, are ordered such that leafs are first, and parents of leafs are next and then the grandparents and so on.

The joint and the marginal distribution are evaluated in the following recursive way
\begin{algorithm}
    \caption*{Calculation of $p(x,y)$}\label{SPN_1}
    \begin{algorithmic}
    \State \textbf{Input:} Fully trained SPN, with leaf distributions $p_i(\cdot)$ for $i\in \mathcal{L}eaf(S)$ and weigts 
    $w_{i,j}$ for $(i,j) \in \{(i,j)|i \in \mathcal{S}um(S), j \in ch(i)\}$ 
    \Function{\text{Eval}}{node i}
    \If{$i \in \mathcal{L}eaf(S)$}
        \State  $\textbf{return: } p_i(x,y)$ \Comment{evaluate leaf distributions at point $(x,y)$}
    \EndIf
    %\For{$i \in I_{o}$}
    \If{$i\in \mathcal{S}um(S)$}
        \State $\textbf{return: } \sum_{j\in ch(i)} w_{i,j} \text{Eval}(j)$
    \EndIf
    \If{$i\in \mathcal{P}rod(S)$}
        \State $\textbf{return: } \prod_{j \in ch(i)} \text{Eval}(j)$
    \EndIf
    \EndFunction
    \State $p(x) =  \text{Eval(root)}$
    \end{algorithmic}
\end{algorithm}

\begin{algorithm}[H]
    \caption*{Calculation of $p(x)$}\label{SPN}
    \begin{algorithmic}
    \State \textbf{Input:} Fully trained SPN, with leaf distributions $p_i(\cdot)$ for all leaves $i$ and weigts $w_{\cdot,\cdot}$ 
    \Function{\text{Eval}}{node i}
    \If{$i \in \mathcal{L}eaf(S)$} %\Comment{leaf node}
        \If{node handle x}
            \State  $\textbf{return: } p_i(x,y)$ \Comment{evaluate leaf distributions at point $(x,y)$}
        \Else 
            \State  $\textbf{return: } 1$ \Comment{set node equal 1 at point $(x,y)$}
        \EndIf
    \EndIf
    \If{$i\in \mathcal{S}um(S)$}
        \State $\textbf{return: } \sum_{j\in ch(i)} w_{i,j} \text{Eval}(j)$
    \EndIf
    \If{$i\in \mathcal{P}rod(S)$}
        \State $\textbf{return: } \prod_{j \in ch(i)} \text{Eval}(j)$
    \EndIf
    \EndFunction
    \State $p(x) =  \text{Eval(root)}$
    \end{algorithmic}
\end{algorithm}

So after doing two slightly different forward passes through the SPN, 
$p(x)$ and $p(x,y)$, using Bayes rule,
we can combined the two queries into the conditional distribution: 
$$p(y|x) = \frac{p(x,y)}{p(x)}$$
The predictive distribution is found with a cost of just $O(E+E+1) = O(E)$, where E is number
of edges/connections in the SPN. 

\subsection{SPN - learning}
It is not enouth to do predictive inference on a SPN, we also need to fit it on
the data. It is possible interpret sum-product network as a large mixture model 
and therefore use expectation-maximization to train the model. 
We will introduce that idea now. The Paper \cite{SPN_EM}... %["Learning Arbitrary Sum-Product Network Leaves
%with Expectation-Maximization"] 
defines SPN as a mixture of all sub-networks of an SPN.

%from [@desana]:
\begin{definition} 
    A sub-network $\bar S_z$ of $S$ is an SPN, which includes the root $S$ and then includes nodes
    according to the following recursive scheme: 
\end{definition}
\begin{algorithm}[H]
    \caption*{Collection of sub-network $S_z$ of $S$}\label{SPN3}
    \begin{algorithmic}
    %\State \textbf{Global:}  $S_z$ 
    \Function{Process}{node i, $S_z$}
    \If{$i \in \mathcal{L}eaf(S)$}
        \State  $\textbf{return: }$ 
    \EndIf
    %\For{$i \in I_{o}$}
    \If{$i\in \mathcal{S}um(S)$}
       %\State $S_z =S_z \cup \{j \in ch(i)\}$ \Comment{include one child of node $i$}
        \State $S_z =S_z.add(j \in ch(i))$ \Comment{include one child of node $i$}
        \State $\textbf{return: } \text{Process}(j, S_z)$
    \EndIf
    \If{$i\in \mathcal{P}rod(S)$}
        \State $S_z =S_z \cup \{j | j \in ch(i)\}$ \Comment{include all childen of node $i$}
        \For{$j \in ch(i)$}
            \State $\textbf{return: } \text{Process}(j,S_z)$
        \EndFor
    \EndIf
    \State $\textbf{return: } S_z$
    \EndFunction
    \State $S_z =  \text{Process(root,Ø)}$
    \end{algorithmic}
\end{algorithm}
So we see that at each sum node the number of different sub-networks multiplies with the number of children for that
sum node. And thereby, the total number of sub-networks is
 $$Z = \prod_{i\in \mathcal{S}um(S)}|ch(i)|$$ 
 i.e. an exponential large amount of sub-networks. This is the amount of
 mixture components implicitly defined in an SPN. 
 Denote the set of edges in the sub-network $\mathcal{E}(S_z)$.
Now the we define a mixture coeficient, $\lambda_z$ and component for each $S_z$ as 
$$\lambda_z := \prod_{(i,j)\in \mathcal{E}(S_z)} w_{i,j}, \hspace{1cm}
p_z(x,y|\theta) := \prod_{i \in \mathcal{L}(S_z)} p_i(x,y)$$
where $p_i(x,y)$ is the leaf distribution at leaf node $i$ paramitised with $\theta$. 
It can now be proven that the SPN can be interpreted as the following mixture model, 
$$p(x,y|w,\theta) = \sum_{z=1}^Z \lambda_z(w)p_z(x,y|\theta)$$
i.e. by the weighted sum of all $Z$ sub-networks. For convinience
we define each sum component as $p(z,x,y|w,\theta) := \lambda_z(w)p_z(x,y|\theta)$.
Evaluation of $p(x,y|w,\theta)$ will never be done as the sum over $Z$ components, 
instead there is a proposition. 

\begin{proposition}
    Consider a SPN, S, a sum node $q \in \mathcal{S}um(S)$ and a child $i \in ch(q)$,
    then the following relation holds, 
    $$\sum_{z:(q,i)\in \mathcal{E}(S_z)} \lambda_z(w) p_z(x,y|\theta) = w_{i,q}
    \frac{\partial S}{\partial v(q)} v(i)$$
\end{proposition}

\newpage
\section{Expectation-maximization for mixture models}
Mixture models can be seen as probibalistic graphical models, <fig> there one mixture component is
picked according to the realization of a catagorical variable $\textbf{Z}$ with parameters according
the the mixture weights, i.e we can reformelate, 
\begin{align*}
    p(x) &= \sum_{k=1}^K w_k p_k(x)\\
   \iff \hspace*{1cm} p(x) &= p_z(x), \hspace*{0.5cm} z \sim Cat(w_i, \dots, w_K).
\end{align*}
In fact $p_z(x)$ is a conditial distribution, $p(x|z)$, and combined with the distribution
of $Z$ we can define the joint 
$$p(x,z):=p_z(x)p(z)$$
The the case of a statistical model, data $\mathcal{D}$ is fitted by the mixture model 
by tuning the model parameters $\theta = \{w, \text{paramers for} p_i\}$. Then the joint
distribtuion $p(\mathcal{D},z| \theta)$ is refered as the \textit{complete-data} likehood in the EM algorithm. 
$$p(\mathcal{D},z|\theta):=p(\mathcal{D}|z,\theta)p(z|\theta)$$ 
When fitting model parameters we essentially want to find the parameters, that maximize the probability of
the parameters given the data, $p(\theta|\mathcal{D})$. Assuming an
unimformative/flat prior $p(\theta)$, 
\begin{align*}
p(\theta|\mathcal{D})&= \frac{p(\mathcal{D}|\theta)p(\theta)}{p(\mathcal{D})}\\
\Rightarrow  \arg\max_{\theta} p(\theta|\mathcal{D}) &= \arg\max_{\theta} p(\mathcal{D}|\theta)
\end{align*}
we arrive at the maximum likelihood estimate (MLE). The task of finding the MLE is conviniently
done using EM algorithm, since we can look at the likehood as the marginalized
\textit{complete-data} likehood, 
$$p(\mathcal{D}|\theta) = \sum_z p(\mathcal{D}, z|\theta)$$

%(Note that this statement true accoring to Theorem 2.1. in \cite{gupta2011theory}). 

\begin{testexample2}[Expectation-maximization EM <based on Bishops book>]
    Expectation maximization is a convinient method for finding ML (or MAP) estimate of a 
    latent variable model. We consider a probibalistic model paramitised with $\theta$, 
    $$p(\textbf{X}, \textbf{Z}|\theta)$$ where we denote all latent variables \textbf{Z}, and
    observed variables \textbf{X}. Our goal is to find the maximum of the likehood, 
    $$p(\textbf{X}|\theta) = \int p(\textbf{X}, \textbf{Z}| \theta) \mu(d\textbf{Z})$$
    maximizating the likehood itself $p(\textbf{X}|\theta)$ is assumed dificult 
    but maximizating of the \textit{complete-data} likehood $p(\textbf{X}, \textbf{Z}|\theta)$
    is much easier. The algorithm iterates over two steps: The expecation (E) step and the maximization (M) step, 
    defined in the following way for iteration $t$, 
    
    \textbf{E-step}

    Define the functional $Q(\theta,\theta^{(t)})$, to be the expected value of the complete-data 
    log likehood (log likehood function of $\theta$), with respect to the only random quantaty $\textbf{Z}$,
    which is assumed to follow a distribtuion with the density $p(\textbf{Z}|\textbf{X}, \theta^{(t)})$,
    i.e. the conditional distribution of \textbf{Z} given \textbf{X} and the current parameter point estimate
    $\theta^{(t)}$: 
    $$Q(\theta,\theta^{(t)}) := E_{p(\textbf{Z}|\textbf{X}, \theta^{(t)})}[\log p(\textbf{X}, \textbf{Z}|\theta)]$$

    \textbf{M-step}

    After the E-step we find the point estimate $\theta^{(t+1)}$ which maximizes $Q(\cdot|\theta^{(t)})$, i.e.
    $$\theta^{(t+1)} = \arg\max_{\theta} Q(\theta|\theta^{(t)})$$

    \begin{algorithm}[H]
        \caption*{(local) maximization of $p(\mathcal{D}|\theta)$}\label{EM}
        \begin{algorithmic}
        \State \textbf{Input:} dataset $\mathcal{D}$, joint model $p(\mathcal{D}, \textbf{Z}|\theta)$
        \While{Not converged}
            \State $Q(\cdot, \theta^{(t)}) \gets E_{p(\textbf{Z}|\mathcal{D}, \theta^{(t)})}[\log p(\mathcal{D}, \textbf{Z}|\theta)]$ \Comment{E-step}
            \State $\theta^{(t+1)} \gets \arg\max_{\theta} Q(\theta|\theta^{(t)})$ \Comment{M-step}
        \EndWhile
        \State $\textbf{return: } \theta^{(end)}$
    \end{algorithmic}
    \end{algorithm}

    \textbf{Proof of correctness} 
    
    We will now give a short proof that maximizing $Q(\cdot|\theta^{(t)})$ maximizes the likelihood
    $p(\textbf{X}|\theta)$, where we assume that $\textbf{Z}$ is a random vector with a discrete
    distribution. This allow us to use Gibbs inequality: 
    $$\sum_z p_1(z) \log p_1(z) \geq \sum_z p_1(z) \log p_2(z)$$ where $p_1(\cdot)$ and $p_2(\cdot)$
    are densities belonging to two discrete distributions of $Z$, equality if $p_1(\cdot) =
    p_2(\cdot)$. From now on we will alter the subscript on the expecations, just have in mind that
    $$E_{\theta^{(t)}}[g(Z)]:=E_{p(\textbf{Z}|\textbf{X}, \theta^{(t)})}[g(Z)] = \sum_z g(z)
    p(\textbf{z}|\textbf{X}, \theta^{(t)})$$ 
    Now to the proof: From bayes rule $p(\textbf{X}|\theta) =
    \frac{p(\textbf{X}, \textbf{Z})}{p(\textbf{X})}$ we can write
    $$\log p(\textbf{X}|\theta) = \log p(\textbf{X}, \textbf{Z}) - \log p(\textbf{Z}|\textbf{X},\theta)$$
    Now, taking the expecation of the above w.r.t. $p(\textbf{Z}|\textbf{X}, \theta^{(t)})$,
    yields,
    \begin{align*}
        \log p(\textbf{X}|\theta)  &= E_{\theta^{(t)}}[\log p(\textbf{X}, \textbf{Z}|\theta)]
        -  E_{\theta^{(t)}}[\log p(\textbf{Z}|\textbf{X},\theta)]\\
        &= Q(\theta,\theta^{(t)})+ E_{\theta^{(t)}}[\log p(\textbf{Z}|\textbf{X},\theta)]
    \end{align*} 
    Since the above equation holds for any $\theta$, it also holds for $\theta^{(t)}$
    now we have, 
    $$\log p(\textbf{X}|\theta^{(t)}) = Q(\theta^{(t)},\theta^{(t)})+ E_{\theta^{(t)}}[\log
    p(\textbf{Z}|\textbf{X},\theta^{(t)})]$$ 
    Subtracting the two equations, we get,  
    $$\log p(\textbf{X}|\theta) - \log p(\textbf{X}|\theta^{(t)}) = Q(\theta,\theta^{(t)})
    -Q(\theta^{(t)},\theta^{(t)})+ E_{\theta^{(t)}}[\log p(\textbf{Z}|\textbf{X},\theta)]-
    E_{\theta^{(t)}}[\log p(\textbf{Z}|\textbf{X},\theta^{(t)})]$$ From Gibb's inequality we have
    that $E_{\theta^{(t)}}[\log p(\textbf{Z}|\textbf{X},\theta^{(t)})]\leq E_{\theta^{(t)}}[\log
    p(\textbf{Z}|\textbf{X},\theta)]$ where equality only holds for $\theta^{(t)} = \theta$, giving
    \begin{align*}
        \log p(\textbf{X}|\theta) - \log p(\textbf{X}|\theta^{(t)}) \geq 
    Q(\theta,\theta^{(t)})
    -Q(\theta^{(t)},\theta^{(t)})
    \end{align*}
    so optimizing $Q(\theta,\theta^{(t)})$ will optimize
    $\log p(\textbf{X}|\theta) $ as least as much.
\end{testexample2}


then we can marginalise over
    $z$ in order to recover $p(x)$, 
    $$p(\textbf{x}|\pi) = \sum_{z=1}^Z p(z,\textbf{x}|\pi) $$
    assuming iid data, then the complete likehood can be decompose as the product 
    $$ p(z,\textbf{x}|\pi) = \prod_{i=1}^n p(z,\textbf{x}_i|\pi)$$
    We wil for convinience transform the likelihood using a log transform, 
    as it will not influence the maximum. 
    $$\log p(\textbf{x}|\pi) = \sum_{z=1}^Z \sum_{i=1}^n \log p(z,\textbf{x}_i|\pi)$$

<EM for SPN >
<EM for GMM >

% and a normal evaluation of $p(x,y|w_{old}, \theta_{old})$
% combined in Bayes rule we obtain
% $$p(z|x,y,w_{old},\theta_{old}) = \frac{p(z,x,y|w_{old},\theta_{old})}{p(x,y|w_{old}, \theta_{old})} =
%  \frac{\lambda_z(w_{old})p_z(x,y|\theta_{old})}{p(x,y|w_{old}, \theta_{old})} $$

% and we have the expecation for the EM-algorithm, 
% $$Q(\pi, \pi_{old}) = \sum_{n=1}^N \sum_{z=1}^Z p(z|xy_n, \pi_{old}) \ln p(z,xy_n|\pi)$$

\begin{tcolorbox}[
    sharp corners,
    boxrule=0mm,
    enhanced,
    borderline west={2pt}{0pt}{red},
    colframe=drGray,
    colback=drGray,
    coltitle=black,
]
{\large \textbf{Her er en titel}}
\begin{itemize}
    \item Lorem ipsum dolor sit amet, consectetur adipiscing elit. Proin euismod finibus enim vel tincidunt. Aliquam a placerat risus. Donec lobortis consequat massa et rhoncus. Cras a quam nec ante porta consequat at in nulla. Praesent sagittis, tortor id iaculis pharetra, dui eros gravida 

\item Cras euismod mauris ut magna porta egestas. Ut non nisl leo. In hac habitasse platea dictumst. Pellentesque sed diam hendrerit tellus sagittis bibendum. Donec lorem augue, aliquet 

 \item In at interdum lacus. Ut purus arcu, consequat a leo at, viverra tincidunt eros. Sed non mi fringilla, ornare velit eget, feugiat tortor. Donec blandit orci at dapibus vehicula. Pellentesque non cursus tellus. Ut et molestie quam. Sed convallis laoreet odio at dignissim. Maecenas condimentum felis eu laoreet pretium. Fusce lacinia ligula purus, non
\end{itemize}
\end{tcolorbox}

\section{Gaussian Mixture Regression}
The Gaussian mixture is a generative model of the joint probability of $x$ and $y$ given as, 
$$p(x,y)= \sum_{k=1}^K \pi^{(k)} \mathcal{N}(x,y|\mu^{(k)},\Sigma^{(k)}), 
\hspace{1cm} \mu=\begin{bmatrix}
    \mu_x \\ \mu_y
\end{bmatrix},\hspace{0.1cm} \Sigma = \begin{bmatrix}
    \Sigma_{xx} & \Sigma_{xy}\\ \Sigma_{yx}& \Sigma_{yy}
\end{bmatrix}$$
This is trained using the EM algorithm. We will now show how the conditial is
calculated exact. 
\subsection{GMR - prediction}
We need the conditional distribution of the Gaussian mixture
in order to get the predictive distribution. We will now formulate
the conditial distribution in terms of conditional and marginals of
the individual mixture components. First of all the marginal distribution $p(x)$ 
of the mixture is given as, \todo{How??!}
$$p(x) = \sum_{k=1}^K \mathcal{N}(x|\mu_{x}^{(k)},\Sigma_{xx}^{(k)})$$ 
next the joint distribution can be decomposed with the probability chain rule,
\begin{align*}
    p(x,y) &= p(x)p(y|x)\\
    \implies \hspace{1cm} \mathcal{N}(x,y|\mu,\Sigma) &= 
    \mathcal{N}(x|\mu_{x},\Sigma_{xx}) \mathcal{N}(y|\mu_{y|x},\Sigma_{y|x})
\end{align*} 
And we can formulate the conditial in terms of individual multivariate Gaussians, 

\begin{align}
    p(y|x) &= \frac{p(y,x)}{p(x)}\\
    &= \sum_{k=1}^K \frac{\pi^{(k)}}{p(x)} \mathcal{N}(x,y|\mu^{(k)},\Sigma^{(k)})\\
    &=  \sum_{k=1}^K \frac{\pi^{(k)} \mathcal{N}(x|\mu_{x}^{(k)},\Sigma_{xx}^{(k)})}{p(x)}\mathcal{N}(y|\mu_{y|x}^{(k)},\Sigma_{y|x}^{(k)})\\
    &=  \sum_{k=1}^K \pi_{y|x}^{(k)} p(y|x,\mu_{y|x}^{(k)},\Sigma_{y|x}^{(k)})
\end{align}

where $\pi_{y|x}^{(k)} := \frac{\pi^{(k)} \mathcal{N}(x|\mu_{x}^{(k)},\Sigma_{xx}^{(k)})}
{\sum_{i=1}^K \mathcal{N}(x|\mu_{x}^{(i)},\Sigma_{xx}^{(i)})}$. So we see that 
the conditonal of a Gaussian mixture model is again a Gaussian mixture model.


\begin{testexample2}[Conditional og multivariate Gaussian]
    The conditional distribution is defined as <ref bishop>
    $$p(y|x,\mu, \Sigma) = \mathcal{N}(y|\mu_{y|x},\Sigma_{y|x} )\hspace{1cm} \mu=\begin{bmatrix}
        \mu_x \\ \mu_y
    \end{bmatrix},\hspace{0.1cm} \Sigma = \begin{bmatrix}
        \Sigma_{xx} & \Sigma_{xy}\\ \Sigma_{yx}& \Sigma_{yy}
    \end{bmatrix}$$ 
    where 
    \begin{align}
        \mu_{y|x} &= \mu_y+\Sigma_{yx}\Sigma_{xx}^{-1}(x-\mu_x)\\
        \Sigma_{y|x} &= \Sigma_{yy}-\Sigma_{yx}\Sigma_{xx}^{-1}\Sigma_{xy} 
    \end{align}
    
\end{testexample2}


\subsection{GMR - Leaning}

EM-algorithm, classic example. 

\section{Gaussian Process Regression}
We now show how the preditive distribution is calculated exact for
Gaussian Processes, i.e. 
\begin{equation}\label{GP_predictive}
    p(y|x,\mathcal{D}) = \int \mathcal{N}(y|f(x), \sigma^2) p(f(x)|\mathcal{D})df(x)
\end{equation}
we will soon see that $p(f(x)|\mathcal{D}) = \mathcal{N}(f(x)| .., ...)$ and
and thereby that we have a marginal Gaussian distribution for $f(x)$ and a 
conditional Gaussian distribution of $y$ given $f(x)$, giving us the marginalized
distribtuion, $p(y|x,\mathcal{D})$, using formulars \eqref{marginal_distribution}. 

\begin{testexample2}[Trick with normal distributions [from Bishops book?]]
    Given a marginal Gaussian distribution of $x$ and a conditional Gaussian distribution
    of $y$ given $x$ of the form, 
    \begin{align*}
        p(x) &= \mathcal{N}(x|\mu, \Lambda^{-1})\\
        p(y|x) &= \mathcal{N}(x|Ax+b, L^{-1})
    \end{align*}
    then the marginal distribution of $y$ and the conditional distribution of $x$ given $y$
    have the form, 
    \begin{align}
        p(y) &= \mathcal{N}(y|A\mu+b,L^{-1}+A \Lambda^{-1}A^T) \label{marginal_distribution}\\
        p(x|y) &= \mathcal{N}(x|\Gamma \mu+\Gamma [A^TL(y-b)],\Gamma )\\
        \Gamma &:= (\Lambda +A^TLA)^{-1}
    \end{align}
\end{testexample2}

\subsection*{Posterior function}
Recall we assume $\textbf{f} = (f(\textbf{x}_1), \dots, f(\textbf{x}_n))$ is the parameters in 
our model, therefore we call $p(\textbf{f}|\mathcal{D})$ the posterior distribution. However, 
what is of real interest is the function values on unobserved locations, thereby we 
extend $\textbf{f}$ to be a function, i.e. an infinitely dimentional vector. We call this 
quantaty \textit{the posterior function} 
\begin{equation}\label{posterior_function}
    p(f(\cdot)|\mathcal{D})= \int p(f(\cdot)|\textbf{x}, \textbf{f})p(\textbf{f}|\mathcal{D})d\textbf{f}.
\end{equation}
% The connection between the two posteriors is given here:
% $$p(f(\cdot)|\mathcal{D}) = \int p(f(\cdot)|\textbf{x}, \textbf{f})p(\textbf{f}|\mathcal{D})d\textbf{f}$$
Prior we assume that the function takes values accoring to
$$p(\textbf{f}|\textbf{x}) = \mathcal{N}(\textbf{f}|\textbf{0}, c(\textbf{x}, \textbf{x}))$$
where the covariance is defined at kernel evaluation for each pair of $\textbf{x}$,
where $c(\cdot, \cdot)$ is a covariance function, chosen to be the Matérn covariance function,

  $$c(\textbf{x}, \textbf{x}) = \begin{bmatrix}
    c(x_1,x_1) & \dots & c(x_1,x_n)\\
    \vdots& \ddots\\
    c(x_n,x_1) & \dots & c(x_n,x_n)
\end{bmatrix}\hspace{1cm} c(x, y) := Matern(x,y)...$$ 

We now calculate the first term in the integral \eqref{posterior_function}, 
$p(f(\cdot)|\textbf{x}, \textbf{f})$ using that we have the joint prior 
distribution, 
\begin{align}
    p(f(\cdot),\textbf{f}|\textbf{x}) = \mathcal{N}\left(\begin{bmatrix}
        f(\cdot)\\ \textbf{f}
    \end{bmatrix} \middle| \begin{bmatrix}
        0\\ \textbf{0}
    \end{bmatrix}, \begin{bmatrix}
        c(\cdot, \cdot) & c(\cdot,\textbf{x})\\
        c(\textbf{x}, \cdot) & c(\textbf{x}, \textbf{x})
    \end{bmatrix} \right)
\end{align}

And the conditonal of a joint Gaussian is given using <ref> 
$$p(f(\cdot)|\textbf{x}, \textbf{f}) = \mathcal{N}(f(\cdot)|c(\cdot, \cdot)^{-1}c(\cdot, \textbf{x})\textbf{f}, c(\cdot, \cdot)^{-1})$$

Next we calculate the last term in the integral \eqref{posterior_function}, 
$p(\textbf{f}|\mathcal{D})$, i.e. the posterior distribution. Assuming iid data, 
i.e. $p(y_1,\dots, y_n|x_1,\dots, x_n, \textbf{f}) = \prod_{i=1}^n p(y_i|x_i,\textbf{f}_i)$
and that the likelihood is Gaussian with mean $\textbf{f}$ and variance $\sigma^2 I_n$. 

\begin{align*}
    p(\textbf{f}|\mathcal{D}) &\propto p(\textbf{f}|x)\prod_{i=1}^n p(y_i|x_i,\textbf{f}_i)\\
    &= \mathcal{N}(\textbf{f}|\textbf{0},c(\textbf{x}, \textbf{x})) \prod_{i=1}^n \mathcal{N}(y|\textbf{f}_i,\sigma^2)\\
    &= \mathcal{N}(\textbf{f}|\textbf{0}, c(\textbf{x}, \textbf{x})) \mathcal{N}(\textbf{y}|\textbf{f},\sigma^2 I_n)
\end{align*}
now from <ref> we have that the posterior is the following Gaussian: 
\begin{equation*}
    p(\textbf{f}|\mathcal{D}) = \mathcal{N}(\textbf{f}|M^{-1} \sigma^{-2}\textbf{y}, M^{-1}) \hspace{0.5cm}M := c(\textbf{x}, \textbf{x})^{-1}+\sigma^{-2} I_n
\end{equation*}
Now we found that both term in the integral \eqref{posterior_function}, and they
are related such that it is possible to use \eqref{marginal_distribution} for arriving 
at (we define $A :=  c(\cdot, \cdot)^{-1} c(\cdot, \textbf{x})$), 
$$p(f(\cdot)|\mathcal{D}) = \mathcal{N}(f(\cdot)|AM^{-1}\sigma^{-2}\textbf{y}, c(\cdot, \cdot)^{-1}+
AM^{-1}A^T)$$

Finally we found that both terms in the integral \eqref{GP_predictive} also is related
in a simlar way, and we use \eqref{marginal_distribution}, again to arrive at the predictive
distribtuion, 
$$p(y_*|x_*,\mathcal{D}) = \mathcal{N}(y_*|AM^{-1}\sigma^{-2}\textbf{y}, c(x_*, x_*)^{-1}+
AM^{-1}A^T+\sigma^2)$$

\todo{Some questions about a naive approach..!}

\subsection*{Learning - Emperical bayes inference}

Anther inference which is done is then optimizing the hyper parameters using emperical bayes i.e.
the variance and length scale for the kernel. Here we optimize the marginalized likelihood function
$$p(y_1, \dots, y_n|x_1, \dots, x_n, \theta) = -\frac{1}{2}[(y-\mu)^T (\Sigma+N)^{-1}(y-\mu)+ \log |\Sigma+N|+n \log 2\pi]$$
<and how to get to there?>

\todo{Model assessment becomes trivial in light of the model posterior if we
simply establish preferences over models according to their posterior
probability. When using the uniform model prior (4.6)
 the model posterior is proportional to the marginal likelihood alone,
which can be then used directly for model assessment. ??! Forstår ikke}

\section{Deep Network Global Optimization - ?} ??
Kernel regression :-))
Should I not work on this?

\section{Bayesian Neural Networks}
Bohamiann and numpyro-BNN are examples of probabilistic models with intractable inference. The predictive density is given as, 
\begin{align*}
    p(y_*|x_*,\mathcal{D}) &= \int p(y_*|x, \theta)p(\theta|\mathcal{D})d\theta\\
    &\approx \frac{1}{K} \sum_{k=1}^K p(y_*|x, \theta^{(k)})
\end{align*}
where the first integral is intractable as $\theta$ can live in a highly dimensional space, second the approximation sign
is true, since we can aproximate the integral with monte carlo sampling: $\theta^{(k)}$ are iid samples from the posterior 
distribution, $\theta^{(k)} \sim p(\theta|\mathcal{D})$. We can get samples from the posterior distriution

\begin{testexample2}[Monte Carlo approximation]
    Assuming we have a number of iid samples, $\theta^{(1)}, \dots, \theta^{(K)}$ drawn from the
    distribution $p(x)$, then the following appriximation 
    $$E[f(x)] \approx \frac{1}{K} \sum_{k=1}^K f(x^{(k)}) =: \Theta_{K}(f)$$
    holds accoring to the law of large numbers 
    in fact $$E[f(x)] = \lim_{K \rightarrow \infty} \Theta_{K}(f)$$
    and the central limit theorem, <OBS refere!>
    $$p(\hat \Theta) \approx \mathcal{N}(\hat \Theta |\mu_f, \frac{\sigma_f^2}{K})$$
    which ensures that the variance of the unbiased estimator of the expecation decreases
    with number of samples, $K$. Left is to sample the $iid$ samples from the distribution $p(x)$
\end{testexample2}
\subsection*{Posterior samples}
For both models 
the joint distribution $p(\mathcal{D},\theta)$ is available, but calulating the posterior distribution requires the
marginalized likehood, $p(\mathcal{D}) = \int_{\theta} p(\mathcal{D},\theta)$. This integral is often intractable
since the space of $\theta$ typically is abnomous - so not even nummerical appriximations of the intergral is tractable.
From Bayes rule, we have the equality, 
$$p(\theta|\mathcal{D}) = \frac{p(\mathcal{D},\theta)}{p(\mathcal{D})} \propto p(\mathcal{D},\theta),$$
where the propotional sign is true, since $p(\theta|\mathcal{D})$ is a function of $\theta$. 
Knowing the $p(\mathcal{D},\theta)$ joint distribution thereby allow for using Markov chain Monte Carlo
for sampling from the posterior distribution.  

\begin{testexample2}[Markov chain Monte Carlo]
    We can conviniently use MCMC for sampling from a probability density $p(x)$, with only the knowledge of a 
    propotional/unnormalised density $\hat p(x) \geq 0$ i.e
    $$\hat p(x) = c\cdot p(x) \hspace{1cm} c = \int \hat p(x) dx,$$
    where $\int \hat p(x) dx$ is a possible intractable integral. 
    An ergodic Markov chain/process is constructed, such that its stationary distribution is exactly $p(x)$, but only
    with the knowledge of $\hat p(x)$. 
\end{testexample2}

\begin{testexample}[Metropolis-Hasting (MH)]
    The most simple MCMC method is the Metropolis-Hasting algorithm. At iteration
    $n$ we have a sample $x_n$,
    \begin{enumerate}
        \item Propose $\hat x$ from a proposal density $q(x_n,\cdot)$
        \item Compute accptance probability $$\alpha(x_n,\hat x) = \min \left(1, \frac{p(\hat x)}{p(x_n)} \frac{q(\hat x, x_n)}{q(x_n,\hat x)}\right)$$
        \item Set the next sample $$x_{n+1} = \begin{cases}
            \hat x &\text{with probability } \alpha(x_n, \hat x)\\
             x_n &\text{with probability } 1-\alpha(x_n, \hat x)
        \end{cases}$$
    \end{enumerate}
    note that $\alpha(x_n,\hat x)$ requires $p(x)$, but since the algorithm only
    requires the fraction $\frac{p(\hat x)}{p(x_n)} = \frac{p(\hat x)\cdot c}{p(x_n)\cdot c} = \frac{\hat p(\hat x)}{\hat p(x_n)}$
    we only need $\hat p$. 
    
    \textbf{Proof:} Assuming discrete states, the transition probability between the states are given as, 
    $$p(x\rightarrow y) = \begin{cases}
        q(x,y)\alpha(x,y) & \text{if } x\neq y\\
        q(x,x) + \sum_{z\neq x} q(x,z)(1-\alpha(x,z)) & \text{if } x=y
    \end{cases}$$
    Now, let us look at the so-called \textit{detailed balance} relation, i.e. that if we are sampling from the
    stationary density we stay there at the next state. Assume $x\neq y$, 
    \begin{align*}
        p(x)p(x\rightarrow y) &= p(x)q(x,y)\alpha(x,y)\\
        &=p(x)q(x,y) \min \left(1, \frac{p(\hat x)}{p(x_n)} \frac{q(\hat x, x_n)}{q(x_n,\hat x)}\right)\\
        &= \min(p(x)q(x,y), p(y)q(y,x))
    \end{align*}
    Observing that the right hand side yields symmetric result in $x$ and $y$, therefore we obtain, 
    $$p(x)p(x\rightarrow y) = p(y)p(y\rightarrow x)$$
    and summing over $x$ on both sides yields,
    \begin{align}
        \sum_x p(x)p(x\rightarrow y) &= p(y) \sum_x p(y\rightarrow x)\\
        \implies \hspace{0.5cm} p(y) &= \sum_x p(x)p(x\rightarrow y)
    \end{align}
    similar conclusion will be obtained for $x = y$, all in all this reveals that $p(x)$ is in fact invariant for the chain
     $\{x_1, \dots , x_n\}$ and thereby that MH is a MCMC algorithm. 
\end{testexample}

% Markov assumption -> history doesn't matter
% Monte Carlo -> Random simulation
% best methods, use gradients. 

% Simulared skate board in a state park. Physic simulation. 
% Often the simulation moves back and fouth and end up in the 
% same point - this is called a U-turn. 

% find global curvature from just knowing the local curvature. 
% Simulation is moving more in the area with high prob. mass. 

% Gradients: Automated differentiation. 

% We want to evaluate integrals of the fom $$E[f(x)] = \int f(x)p(x)dx$$ where $x \in \mathbb{R}^n$ is a
% random vector under the distribution $p(x)$. we are interested in problems where the form of $f(x)$ or $p(x)$
% makes the integral intractalbe. 

% \begin{testexample}[Bayesian neural network]
%     Choosing $f := \mathcal{N}(y;NN_w(x), \sigma)$ and looking at $\theta := (w,\sigma)$ as the random quantaty
%     under the posterior distribution $p(\theta|\mathcal{D})$ we indeed have case of a intractalbe expectation
% \end{testexample}



% Transition density kernel (transtion matrix for finite discrete spaces) $p(x^{(k)}|x^{(k-1)})$



% c) the convergence rate is independent on the
% dimensionality, l. The latter property is in contrast to methods based on the deterministic numerical
% integration, which, in general, have a rate of convergence that slows down as the dimensionality
% increases.

HM with random walk transition is very simple and it comes with some serious disadvantages:
slow convergence speed, might stay in the same region for a long time and produces highly correlated sampels. 
We can do better by replacing the random walk with gradient-guided movements and intepretate the probability
landscape as a physical system.


\begin{testexample}[HMC]
The golden standard in MCMC is the Hamilton monte carlo, which exploits arguments from classical mechanics
around the Hamiltonian equations. This method leads to more efficient sampling as the Hamiltonian intepretation allows the system
to take regions with high probability mass into acound - this is optained using gradient
of the probability landscape.  $\frac{-\partial E(x)}{\partial x} $
 
We define PDF
$$p(x) = \frac{1}{Z_E}\exp(-E(x)),$$

where $E(x)$ is interpretted as the systems potential energy. Now, an latent vector $q$ is introduced in order
to represent the momentum of the system, which gives us the kinetic energy of the system. 

$$K(q) = \frac{1}{2}\sum_{i=1}^l q_i^2$$

Giving the Hamilton function and its coresponing distribution

$$H(x,q)= E(x)+K(q)$$

and 
\begin{align}
    p(x,q) &= \frac{1}{Z_H} \exp(-H(x,q))\\
    &= \frac{1}{Z_E} \exp(-E(x))\frac{1}{Z_K} \exp(-K(x))\\
    &= p(x)p(q)
\end{align}

The desired distribution $p(x)$ is found as the marginal of $p(x,q)$

\end{testexample}

since some intuition is now established around Hamiltonian Monte Carlo, 
we look a bit on the two versions used in Numpyro-BNN and Bohamiann, 

\subsection{No U-Turn sampling}

othen the physical simulation in HMC goes forth and back the same path, and we risk getting bad samples.
No U-turn (NUTS) sampling avoid this. 

\subsection{Adaptive stochatic HMC}
... 



\newpage
\section{Appendix - SPN}
Normalization of SPN
\begin{algorithm}
    \caption*{Calculation of $Z$}\label{SPN2}
    \begin{algorithmic}
    \State \textbf{Input:} node $i$ %with leaf distributions $p_j(\cdot)$ for all $j\in Leafs$ and weigts $w_{\cdot,\cdot}$ 
    \Function{Eval}{node i}
    \If{$i$ is leaf} \Comment{leaf node}
        \State  $\textbf{return: } 1$ \Comment{set node equal 1 at point $(x,y)$}
    \EndIf
    %\For{$i \in I_{o}$}
    \If{$i\in \mathcal{S}$}
        \State $\textbf{return: } \sum_{j\in ch(i)} w_{i,j} \text{Eval}(j)$
    \EndIf
    \If{$i\in \mathcal{P}$}
        \State $\textbf{return: } \prod_{j \in ch(i)} \text{Eval}(j)$
    \EndIf
    \EndFunction
    \State $Z =  \text{Eval(root node)}$
    \end{algorithmic}
\end{algorithm}