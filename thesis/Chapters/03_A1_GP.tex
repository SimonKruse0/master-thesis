Assume we have a multivariate normal distribtuion with mean 0 and variance maxtric, 
$$\begin{bmatrix}
    1&c \\ 
    c&1
\end{bmatrix}$$
Now assume this 


It "tilted"! Because of the positive covariance value between $f_1$ and $f_2$, when $f_1$ increases
in value, the probability of $f_2$ also increasing is higher. So the practical effect is that the
probability along the $f_1=f_2$ line increases, thus leading to the "rotated" shape that we see
above.

So far this is rather obvious, right? But what about higher dimensions? The same properties apply,
but we can't visualize the result in the same way anymore. How can we visualize a sample from a
multivariate Gaussian with 50 dimensions? 

Consider the case where the values of $\textbf{f} = (f_1,f_2,\dots,f_N)^T$ correspond to the
evaluations of a given function $f(x)$ at equidistant points $x_1,x_2,\dots,x_N$ (we will relax this
equidistant assumption later). For example, supose that $\textbf{f} = (f(1),f(2),\dots,f(N))^T$. We
can model these with a N-dimensional multivariate Gaussian! In that case, a rather obvious way for
visualizing the samples from that N-dimensional multivariate Gaussian is simply to plot them in 2-D.
The x-axis holdes the values of $x_i$, and the y-axis represents the correponding sampled value $f_i
= f(x_i)$. 

Let us now create a 50-dimensional multivariate Gaussian and do just that! We need to specify the
mean vector $\boldsymbol\mu$ and the covariance matrix $\boldsymbol\Sigma$. The mean vector we can
just that is a vector zeros, $\boldsymbol\mu = \textbf{0}$. As for the covariance matrix
$\boldsymbol\Sigma$, we will make it structured, such that values close to the diagonal have higher
values. Namely, the covariance between $f_i$ and $f_{i+1}$ is 0.5, the covariance between $f_i$ and
$f_{i+2}$ is 0.25, the covariance between $f_i$ and $f_{i+3}$ is 0.125, ... 