\subsection{Expectation-maximization for mixture models}
Mixture models can be seen as probibalistic graphical models, <fig> there one mixture component is
picked according to the realization of a catagorical variable $\textbf{Z}$ with parameters according
the the mixture weights, i.e we can reformelate, 
\begin{align}\label{mixture_pdf}
    p(x) &= \sum_{k=1}^K w_k p_k(x)\\
   \iff \hspace*{1cm} p(x) &= p_z(x), \hspace*{0.5cm} z \sim Cat(w_i, \dots, w_K).
\end{align}
In fact $p_z(x)$ is a conditial distribution, $p(x|z)$, and combined with the distribution
of $Z$ we can define the joint 
$$p(x,z):=p_z(x)p(z)$$
The the case of a statistical model, data $\mathcal{D}$ is fitted by the mixture model 
by tuning the model parameters $\theta = \{w, \text{paramers for} p_i\}$. Then the joint
distribtuion $p(\mathcal{D},z| \theta)$ is refered as the \textit{complete-data} likehood in the EM algorithm. 
$$p(\mathcal{D},z|\theta):=p(\mathcal{D}|z,\theta)p(z|\theta)$$ 
When fitting model parameters we essentially want to find the parameters, that maximize the probability of
the parameters given the data, $p(\theta|\mathcal{D})$. Assuming an
unimformative/flat prior $p(\theta)$, 
\begin{align*}
p(\theta|\mathcal{D})&= \frac{p(\mathcal{D}|\theta)p(\theta)}{p(\mathcal{D})}\\
\Rightarrow  \arg\max_{\theta} p(\theta|\mathcal{D}) &= \arg\max_{\theta} p(\mathcal{D}|\theta)
\end{align*}
we arrive at the maximum likelihood estimate (MLE). The task of finding the MLE is conviniently
done using EM algorithm, since we can look at the likehood as the marginalized
\textit{complete-data} likehood, 
$$p(\mathcal{D}|\theta) = \sum_z p(\mathcal{D}, z|\theta)$$

%(Note that this statement true accoring to Theorem 2.1. in \cite{gupta2011theory}). 

\begin{testexample2}[Expectation-maximization EM <based on \cite{bishop}]
    Expectation maximization is a convinient method for finding ML (or MAP) estimate of a 
    latent variable model. We consider a probibalistic model paramitised with $\theta$, 
    $$p(\textbf{X}, \textbf{Z}|\theta)$$ where we denote all latent variables \textbf{Z}, and
    observed variables \textbf{X}. Our goal is to find the maximum of the likehood, 
    $$p(\textbf{X}|\theta) = \int p(\textbf{X}, \textbf{Z}| \theta) \mu(d\textbf{Z})$$
    maximizating the likehood itself $p(\textbf{X}|\theta)$ is assumed dificult 
    but maximizating of the \textit{complete-data} likehood $p(\textbf{X}, \textbf{Z}|\theta)$
    is much easier. The algorithm iterates over two steps: The expecation (E) step and the maximization (M) step, 
    defined in the following way for iteration $t$, 
    
    \textbf{E-step}

    Define the functional $Q(\theta,\theta^{(t)})$, to be the expected value of the complete-data 
    log likehood (log likehood function of $\theta$), with respect to the only random quantaty $\textbf{Z}$,
    which is assumed to follow a distribtuion with the density $p(\textbf{Z}|\textbf{X}, \theta^{(t)})$,
    i.e. the conditional distribution of \textbf{Z} given \textbf{X} and the current parameter point estimate
    $\theta^{(t)}$: 
    $$Q(\theta,\theta^{(t)}) := E_{p(\textbf{Z}|\textbf{X}, \theta^{(t)})}[\log p(\textbf{X}, \textbf{Z}|\theta)]$$

    \textbf{M-step}

    After the E-step we find the point estimate $\theta^{(t+1)}$ which maximizes $Q(\cdot|\theta^{(t)})$, i.e.
    $$\theta^{(t+1)} = \arg\max_{\theta} Q(\theta|\theta^{(t)})$$

    \begin{algorithm}[H]
        \caption*{(local) maximization of $p(\mathcal{D}|\theta)$}\label{EM}
        \begin{algorithmic}
        \State \textbf{Input:} dataset $\mathcal{D}$, joint model $p(\mathcal{D}, \textbf{Z}|\theta)$
        \While{Not converged}
            \State $Q(\cdot, \theta^{(t)}) \gets E_{p(\textbf{Z}|\mathcal{D}, \theta^{(t)})}[\log p(\mathcal{D}, \textbf{Z}|\cdot)]$ \Comment{E-step}
            \State $\theta^{(t+1)} \gets \arg\max_{\theta} Q(\theta|\theta^{(t)})$ \Comment{M-step}
        \EndWhile
        \State $\textbf{return: } \theta^{(end)}$
    \end{algorithmic}
    \end{algorithm}

    \textbf{Proof of correctness} 
    
    We will now give a short proof that maximizing $Q(\cdot|\theta^{(t)})$ maximizes the likelihood
    $p(\textbf{X}|\theta)$, where we assume that $\textbf{Z}$ is a random vector with a discrete
    distribution. This allow us to use Gibbs inequality: 
    $$\sum_z p_1(z) \log p_1(z) \geq \sum_z p_1(z) \log p_2(z)$$ where $p_1(\cdot)$ and $p_2(\cdot)$
    are densities belonging to two discrete distributions of $Z$, equality if $p_1(\cdot) =
    p_2(\cdot)$. From now on we will alter the subscript on the expecations, just have in mind that
    $$E_{\theta^{(t)}}[g(Z)]:=E_{p(\textbf{Z}|\textbf{X}, \theta^{(t)})}[g(Z)] = \sum_z g(z)
    p(\textbf{z}|\textbf{X}, \theta^{(t)})$$ 
    Now to the proof: From bayes rule $p(\textbf{X}|\theta) =
    \frac{p(\textbf{X}, \textbf{Z})}{p(\textbf{X})}$ we can write
    $$\log p(\textbf{X}|\theta) = \log p(\textbf{X}, \textbf{Z}) - \log p(\textbf{Z}|\textbf{X},\theta)$$
    Now, taking the expecation of the above w.r.t. $p(\textbf{Z}|\textbf{X}, \theta^{(t)})$,
    yields,
    \begin{align*}
        \log p(\textbf{X}|\theta)  &= E_{\theta^{(t)}}[\log p(\textbf{X}, \textbf{Z}|\theta)]
        -  E_{\theta^{(t)}}[\log p(\textbf{Z}|\textbf{X},\theta)]\\
        &= Q(\theta,\theta^{(t)})+ E_{\theta^{(t)}}[\log p(\textbf{Z}|\textbf{X},\theta)]
    \end{align*} 
    Since the above equation holds for any $\theta$, it also holds for $\theta^{(t)}$
    now we have, 
    $$\log p(\textbf{X}|\theta^{(t)}) = Q(\theta^{(t)},\theta^{(t)})+ E_{\theta^{(t)}}[\log
    p(\textbf{Z}|\textbf{X},\theta^{(t)})]$$ 
    Subtracting the two equations, we get,  
    $$\log p(\textbf{X}|\theta) - \log p(\textbf{X}|\theta^{(t)}) = Q(\theta,\theta^{(t)})
    -Q(\theta^{(t)},\theta^{(t)})+ E_{\theta^{(t)}}[\log p(\textbf{Z}|\textbf{X},\theta)]-
    E_{\theta^{(t)}}[\log p(\textbf{Z}|\textbf{X},\theta^{(t)})]$$ 
    From Gibb's inequality we have
    that $E_{\theta^{(t)}}[\log p(\textbf{Z}|\textbf{X},\theta^{(t)})]\leq E_{\theta^{(t)}}[\log
    p(\textbf{Z}|\textbf{X},\theta)]$ where equality only holds for $\theta^{(t)} = \theta$, giving
    \begin{align*}
        \log p(\textbf{X}|\theta) - \log p(\textbf{X}|\theta^{(t)}) \geq 
    Q(\theta,\theta^{(t)})
    -Q(\theta^{(t)},\theta^{(t)})
    \end{align*}
    so optimizing $Q(\theta,\theta^{(t)})$ will optimize
    $\log p(\textbf{X}|\theta) $ as least as much.
\end{testexample2}
\newpage
 \subsection{EM for Gaussian mixture}
 For a Gaussian mixture the $p_k$ distributions in \eqref{mixture_pdf} is substituted by Gaussain pdfs, 
 i.e. $p_k(x) = \mathcal{N}(x| \mu_k , \Sigma_k)$
 and the density of a catagorical distribution is $p(z) = \sum_{k=1}^K 1_{z=k} w_k = w_z$, combining the two
 we get the joint distribution, 
 $$p(x,z|w,\mu , \Sigma) = w_z \mathcal{N}(x| \mu_z , \Sigma_z)$$

 Taking the log and defning $\theta = \{\mu_1 , \Sigma_1,w_1, \dots, \mu_K , \Sigma_K, w_K\}$, and assuming iid data
 $$\log p(X,Z| \theta) = \sum_{i}^n \left(\log(w_{z_i})+\log(\mathcal{N}(x_i| \mu_{z_i} , \sigma_{z_i}))\right)$$

Now we are ready to calculate $Q(\theta, \theta^{(t)})$, by taking the expectation of the complete-data log likehood
with respect to the distribution, $p(Z|X, \theta)$, 
\begin{align*}
    E_{p(Z|X,  \theta^{(t)})}[\log p(X,Z| \theta)] &=\sum_{i}^n  E_{p(Z|X,  \theta^{(t)})}[p(X_i,Z_i| \theta) ]\\
    &=\sum_{i}^n  E_{p(z_i|x_i,  \theta^{(t)})}[p(x_i,z_i| \theta) ]
\end{align*}

\begin{tcolorbox}[
    sharp corners,
    boxrule=0mm,
    enhanced,
    borderline west={4pt}{0pt}{gray},
    colframe=drGray,
    colback=drGray,
    coltitle=black,
]
{\large \textbf{Expectation with repect to unnecessary variables}}\\
the last equation holds since taking expecation over a function of a random variable $x$ with respect to a that random
variable and more random variabels, $x,y$, is equivalent to the expecation with respect to just $x$, i.e. 
\begin{align*}
    E_{x,y}[g(x)] &= \int\int g(x) p(x,y) dy dx\\
     &= \int g(x) \int p(x,y) dy dx \\
    &= \int g(x) p(x) dy = E_x[g(x)]
\end{align*}
\end{tcolorbox}

the posterior distribution is calculated the following way,
\begin{align*}
    p(z|x, \theta^{(t)}) &= \frac{p(x,z|\theta^{(t)})}{p(x|\theta^{(t)})} \\
    &= \frac{p(x,z|\theta^{(t)})}{\sum_{z} p(x,z|\theta^{(t)})}\\
    &= \frac{w_z^{(t)} \mathcal{N}(x|\mu_z^{(t)}, \Sigma_z^{(t)})}{\sum_{k=1}^K w_k^{(t)} 
    \mathcal{N}(x|\mu_k^{(t)}, \Sigma_k^{(t)})}
\end{align*}
For simplificaiton we will denote, $\gamma^{(t)}(z_i) := p(z_i|x_i, \theta^{(t)})$, 
interpreted as the probability of datapoint $x_i$ to belong to class $z_i$. Bishop \cite*{bishop}
calls this probability function the \textit{responsibility}. We can now conclude the \textbf{E-step}.

\begin{align*}
    Q(\cdot, \theta^{(t)}) &= \sum_{i=1}^n p(x_i,z_i|\cdot) \gamma^{(t)}(z_i) \\
    &= \sum_{i=1}^n \left[ \gamma^{(t)}(z_i) \log(\cdot_{z_i})+\gamma^{(t)}(z_i) 
    \log(\mathcal{N}(x_i| \cdot_{z_i} , \cdot_{z_i}))\right]
\end{align*}
or more concretely $\theta = \{\mu_1 , \Sigma_1,w_1, \dots, \mu_K , \Sigma_K, w_K\}$, 
$$ Q(\theta, \theta^{(t)}) = \sum_{i=1}^n \gamma^{(t)}(z_i) \log(w_{z_i})+\gamma^{(t)}(z_i) \log(\mathcal{N}(x_i| \mu_{z_i} , \Sigma_{z_i})).$$
$Q(\cdot, \theta^{(t)})$ is a concave function - the Gaussian is log-concave and a sum of concave functions is 
also concave - so it is sufficient and nessesary to find its maxima by the root of its derivative, 
$$\frac{d}{d \theta} Q(\theta^*, \theta^{(t)}) = 0 \iff \theta^* = \arg\max_{\theta} Q(\theta, \theta^{(t)})$$

Giving the updates...

\subsection*{EM for SPN}


% and a normal evaluation of $p(x,y|w_{old}, \theta_{old})$ combined in Bayes rule we obtain
% $$p(z|x,y,w_{old},\theta_{old}) = \frac{p(z,x,y|w_{old},\theta_{old})}{p(x,y|w_{old},
% \theta_{old})} = \frac{\lambda_z(w_{old})p_z(x,y|\theta_{old})}{p(x,y|w_{old}, \theta_{old})} $$

% and we have the expecation for the EM-algorithm, 
% $$Q(\pi, \pi_{old}) = \sum_{n=1}^N \sum_{z=1}^Z p(z|xy_n, \pi_{old}) \ln p(z,xy_n|\pi)$$